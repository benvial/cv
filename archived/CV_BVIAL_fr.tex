%%%%%%%%%%%%%%%%%%%%%%%%%%%%%%%%%%%%%%%%%
% Friggeri Resume/CV
% XeLaTeX Template
% Version 1.0 (5/5/13)
%
% This template has been downloaded from:
% http://www.LaTeXTemplates.com
%
% Original author:
% Adrien Friggeri (adrien@friggeri.net)
% https://github.com/afriggeri/CV
%
% License:
% CC BY-NC-SA 3.0 (http://creativecommons.org/licenses/by-nc-sa/3.0/)
%
% Important notes:
% This template needs to be compiled with XeLaTeX and the bibliography, if used,
% needs to be compiled with biber rather than bibtex.
%
%%%%%%%%%%%%%%%%%%%%%%%%%%%%%%%%%%%%%%%%%

\documentclass[]{cv} % Add 'print' as an option into the square bracket to remove colors from this template for printing


\geometry{left=6.1cm,top=1.8cm,right=1.5cm,bottom=0.7cm,nohead,nofoot}

\def\firstname{Benjamin}
\def\familyname{Vial}
\def\FileSubject{curriculum vitae}
\def\FileAuthor{\firstname~\familyname}
\def\FileTitle{\FileSubject~de~\firstname~\familyname}
\def\FileKeyWords{\firstname~\familyname, \FileSubject}



  \RequirePackage[unicode]{hyperref}% unicode is required for unicode pdf metadata
  \hypersetup{
    breaklinks,
    baseurl       = http://,
    pdfborder     = 0 0 0,
    pdfpagemode   = UseNone,% do not show thumbnails or bookmarks on opening
    pdfstartpage  = 1,
%    pdfproducer   = {\LaTeX{}},% will/should be set automatically to the correct TeX engine used
    bookmarksopen = true,
    bookmarksdepth= 2,% to show sections and subsections
    pdfauthor   = {\FileAuthor},%
    pdftitle    = {\FileTitle},%
    pdfsubject  = {\FileSubject},%
    pdfkeywords = {\FileKeyWords},%
    pdfcreator  = {\LaTeX},%
    pdfproducer = {\LaTeX}
    }


\addbibresource{publications.bib} % Specify the bibliography file to include publications

\begin{document}

\header{Benjamin}{~Vial}{Docteur Ingénieur | Optique et Photonique} % Your name and current job title/field

%----------------------------------------------------------------------------------------
%	SIDEBAR SECTION
%----------------------------------------------------------------------------------------

\begin{aside} % In the aside, each new line forces a line break
\section{Contact}
76 quartier le Clos
13360 Roquevaire, France
\faPhone~~+44~7840~029~744~~/~~+33~630~609~609
\faEnvelope~~\href{mailto:b.vial@qmul.ac.uk}{b.vial@qmul.ac.uk}
\faUser~~\href{www.benjaminvial.org}{benjaminvial.org}
~
\faPhone~~+33~630~609~609
\section{Informations}
né le 09/11/1984
nationalité française
\section{Langues}
français : langue maternelle
anglais : courant
espagnol : bases
\section{Programmation}
\textbf{systèmes d'exploitation}
Linux, Windows
\textbf{languages et scripts}
C, C++, Matlab, Python, Mathematica, \LaTeX
\textbf{applications}
Comsol Multiphysics, Gmsh, GetDP, Gsolver, Gimp, LibreOffice
\section{Intérêts}
\textbf{professionels}
photonique
processus résonants
interaction lumière matière
méthodes numériques
méthode des éléments finis
analyse modale
fabrication
caracterisation
\textbf{personnels}
jouer de la guitare
musique
football
snowboard
marche à pied
voyages
cuisine
\end{aside}


%----------------------------------------------------------------------------------------
%	EDUCATION SECTION
%----------------------------------------------------------------------------------------

\section{Formation}

\begin{entrylist}
%------------------------------------------------
\entry
{Apr. 2013}
{Thèse de doctorat {\normalfont en Physique}}
{\href{http://www.fresnel.fr/spip/}{Institut Fresnel}, CNRS, Centrale Marseille, Aix Marseille Universit\'e, Marseille, France}
{Optique, Photonique et traitement d'image}
%------------------------------------------------
\entry
{Oct. 2009}
{Master {\normalfont en Physique}}
{\href{http://www.centrale-marseille.fr/}{Centrale Marseille}~--~
\href{http://www.lma.cnrs-mrs.fr/}{Laboratoire de Mécanique et d'Acoustique}, CNRS, Marseille, France}
{Mécanique, Physique et Ingéniérie, specialisation en Acoustique}

\entry
{Oct. 2009}
{Diplôme d'{Ingénieur Généraliste}}
{\href{http://www.centrale-marseille.fr/}{Centrale Marseille}, Marseille, France}
{}
%------------------------------------------------
\end{entrylist}

%----------------------------------------------------------------------------------------
%	WORK EXPERIENCE SECTION
%----------------------------------------------------------------------------------------
\vspace*{-0.2cm}
\section{Activités de recherche}

\begin{entrylist}


\entry
{depui\\Jan. 2017}
{Postdoctorat}
{\href{http://antennas.eecs.qmul.ac.uk/}{Queen Mary, University of London}, Londres, RU}
{Projet AOTOMAT: Outils d'optimisation pour la conception de matériaux et composants electromagnétiques.
}

\entry
{Jul. 2014 \\Dec. 2016}
{Postdoctorat}
{\href{http://www.eecs.qmul.ac.uk}{Queen Mary University of London}, Londres, RU}
{Projet Quest for Ultimate Electromagnetics Using Spatial Transformations (QUEST).
Optique de transformation appliquée à la conception, l'analyse et la caractérisation de composants à base de métamatériaux.
}


%------------------------------------------------
\entry
{Nov. 2013\\Jan. 2014}
{Postdoctorat}
{\href{http://www.fresnel.fr/spip/}{Institut Fresnel}, Marseille, France}
{Antennes résonantes. \'Etude numérique du couplage entre lumière et particules sub longueur d'onde.
Analyse modale des résonances électriques et magnétiques pour contrôler l'émission et la densité locale d'états.
}
%------------------------------------------------
\entry
{Mai 2013 \\Oct. 2013}
{Postdoctorat}
{\href{http://www.fresnel.fr/spip/}{Institut Fresnel}, Marseille, France}
{Développement d'outils de simulation pour le tracé de rayons en milieu complexe.
}

%------------------------------------------------
\entry
{Oct. 2009\\Avr. 2013}
{Thèse de Doctorat en Physique}
{\href{http://www.fresnel.fr/spip/}{Institut Fresnel}~--~\href{http://www.silios.com/}{Silios Technologies}, Marseille, France}
{\emph{\'Etude de résonateurs électromagnétiques ouverts par approche modale.
Application au filtrage multispectral dans l'infrarouge.(mi-temps université/entreprise)}\\
Modèles numérique part éléments finis pour le calcul des modes propres de
 l'opérateur de Maxwell pour des structures ouvertes.
Représentation modale du champ diffracté donnant par un modèle réduit et conditions
d'excitation résonante des modes à fuites.
  Application à la conception, fabrication et caractérisation de
  filtres diffractifs dans l'infrarouge pour des imageurs multispectraux: passe bande en
   transmission et coupe bande en réflexion basés sur des métamatériaux.
}


%------------------------------------------------

\entry
{Avr. 2009 \\Sept. 2009}
{Stage, Master recherche}
{\href{http://www.lma.cnrs-mrs.fr/}{Laboratoire de Mécanique et d'Acoustique, CNRS, Marseille, France}}
{\emph{Réduction de modèle d'instrument à anche pour la synthèse sonore : un approche par modes propres orthogonaux.}\\
\'Etude numérique et validation de la décomposition du champ de pression
sur les modes propres orthogonaux, optimale au sens de la distribution d'énergie.
}


\end{entrylist}

%----------------------------------------------------------------------------------------
%	Teaching/supervising experience SECTION
%----------------------------------------------------------------------------------------
\vspace*{-0.2cm}
\section{Expérience d'enseignement/encadrement}

\begin{entrylist}
%------------------------------------------------
\entry
{2012}
{Tuteur de stage}
{{Institut Fresnel}, CNRS, Centrale Marseille, Marseille, France}
{Optimisation de filtres spectraux diffractifs (1 élève ingénieur, 3 mois).}

%------------------------------------------------
\entry
{2011}
{Tuteur de stage}
{{Institut Fresnel}, CNRS, Centrale Marseille, Marseille, France}
{Optimisation de l'absorption dans des cellules solaires (4 élèves ingénieur, 1 mois).}

%------------------------------------------------

\end{entrylist}

\vspace*{-0.2cm}
\section{Prix et récompenses}

{Best PhD thesis in 2013 award} from the {\href{http://ed352.sciences.univmed.fr/}{Doctoral School 352, Physics and Condensed Matter Science}}

{Best PhD thesis in 2013 award from {\href{www.cnano-paca.fr/}{CNano PACA}}, finalized research category}

%------------------------------------------------------------------
%
\newgeometry{left=2cm,top=1cm,right=1.5cm,bottom=1.cm,nohead,nofoot}

 \section{Publications}
% % sorting=ydnt,
 \printbibsection{article}{Articles dans des revues internationales à comité de lecture} % Print all articles from the bibliography
%

% \printbibsection{inproceedings}{international peer-reviewed conferences/proceedings}

%
\begin{refsection} %
\nocite{*}
\printbibliography[sorting=chronological, type=inproceedings, title={Conférences internationales avec actes}, keyword={lecture comittee}, heading=subbibliography]
\end{refsection}

\printbibsection{book}{Contribution à un chapitre de livre} % Print all books from the bibliography
% 
%
%
% \begin{refsection} %
% \nocite{*}
% \printbibliography[sorting=chronological, type=inproceedings, title={Conférences internationales sans actes}, keyword={conference}, heading=subbibliography]
% \end{refsection}
%
%
% \begin{refsection}
% \nocite{*}
% \printbibliography[sorting=chronological, type=inproceedings, title={Conférences nationales et séminaires}, keyword={france}, heading=subbibliography]
% \end{refsection}
%
% \begin{refsection}
% \nocite{*}
% \printbibliography[sorting=chronological, type=misc, title={En préparation}, keyword={preparation}, heading=subbibliography]
% \end{refsection}


\end{document}
