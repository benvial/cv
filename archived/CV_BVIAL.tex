%%%%%%%%%%%%%%%%%%%%%%%%%%%%%%%%%%%%%%%%%

% Important notes:
% This template needs to be compiled with XeLaTeX and the bibliography, if used,
% needs to be compiled with biber rather than bibtex.
%
%%%%%%%%%%%%%%%%%%%%%%%%%%%%%%%%%%%%%%%%%

\documentclass[]{cv} % Add 'print' as an option into the square bracket to remove colors from this template for printing


\geometry{left=6.1cm,top=1.8cm,right=1.5cm,bottom=0.7cm,nohead,nofoot}

\def\firstname{Benjamin}
\def\familyname{Vial}
\def\FileSubject{curriculum vitae}
\def\FileAuthor{\firstname~\familyname}
\def\FileTitle{\firstname~\familyname's~\FileSubject}
\def\FileKeyWords{\firstname~\familyname, \FileSubject}



  \RequirePackage[unicode]{hyperref}% unicode is required for unicode pdf metadata
  \hypersetup{
    breaklinks,
    baseurl       = http://,
    pdfborder     = 0 0 0,
    pdfpagemode   = UseNone,% do not show thumbnails or bookmarks on opening
    pdfstartpage  = 1,
    bookmarksopen = true,
    bookmarksdepth= 2,% to show sections and subsections
    pdfauthor   = {\FileAuthor},%
    pdftitle    = {\FileTitle},%
    pdfsubject  = {\FileSubject},%
    pdfkeywords = {\FileKeyWords},%
    pdfcreator  = {\LaTeX},
    }

%    pdfproducer   = {\LaTeX{}},% will/should be set automatically to the correct TeX engine used

% \usepackage{wrapfig}


\addbibresource{publications.bib} % Specify the bibliography file to include publications

\begin{document}

\header{Benjamin}{~Vial}{PhD Engineer | Microwave Engineering and Photonics} % Your name and current job title/field

%----------------------------------------------------------------------------------------
%	SIDEBAR SECTION
%----------------------------------------------------------------------------------------

\begin{aside} % In the aside, each new line forces a line break
\section{Contact}
\faHome 146 Glyn road
London E5 0JE, UK
\faPhone~~+44~7840~029~744
\faEnvelope~~\href{mailto:b.vial@qmul.ac.uk}{b.vial@qmul.ac.uk}
\faUser~~\href{http://bvial.info/}{bvial.info}
\section{Information}
date of birth 09/11/1984
French citizenship
\section{Languages}
French mother tongue
English fluent
Spanish basic
\section{Programming}
\textbf{operating systems}
Linux, Windows
\textbf{languages and scripts}
Python, Matlab, Mathematica, \LaTeX, C, C++, HTML, CSS
\textbf{applications}
Comsol Multiphysics, Gmsh, GetDP, Fenics, Gimp, LibreOffice, Labview
\section{Interests}
\textbf{professional}
microwave engineering
Photonics
Transformation Optics
invisibility cloaking
resonant processes
light-matter interraction
plasmonics
computational EM
numerical modelling
optimization techniques
inverse design
machine learning
finite element method
Fourier modal method
FDTD
modal analysis
wave physics
fabrication
characterization
open source science
\textbf{personal}
playing the guitar
listening to music
football, snowboard, hiking
traveling, cooking
\end{aside}


%----------------------------------------------------------------------------------------
%	EDUCATION SECTION
%----------------------------------------------------------------------------------------

\section{Education}

\begin{entrylist}
%------------------------------------------------
\entry
{Apr. 2013}
{PhD {\normalfont in Physics}}
{\href{http://www.fresnel.fr/}{Institut Fresnel}, CNRS, Centrale Marseille, Aix Marseille Universit\'e, Marseille, France}
{Optics, Photonics and image processing}
%------------------------------------------------
\entry
{Oct. 2009}
{Master's degree {\normalfont in Physics}}
{\href{http://www.centrale-marseille.fr/}{~~Centrale Marseille}~/~
\href{http://www.lma.cnrs-mrs.fr/}{Laboratoire de Mécanique et d'Acoustique}, CNRS, Marseille, France}
{Mechanics, Physics and Engineering, specialization in Acoustics}

\entry
{Oct. 2009}
{Master's degree {\normalfont in Engineering}}
{\href{http://www.centrale-marseille.fr/}{Centrale Marseille}, Marseille, France}
{High level scientific and technical training}
%------------------------------------------------
\end{entrylist}

%----------------------------------------------------------------------------------------
%	WORK EXPERIENCE SECTION
%----------------------------------------------------------------------------------------
\vspace*{-0.2cm}
\section{Research activities}

\begin{entrylist}
%------------------------------------------------



\entry
{Jan. 2019 \\Now}
{Postdoctoral Research Assistant}
{\href{http://antennas.eecs.qmul.ac.uk/}{Queen Mary, University of London}, London, UK}
{Project ANIMATE: nonlinear coupling model and homogenization of ferroelectric 
metamaterials and inverse electromagnetic design.
}


\entry
{Jan. 2017 \\Dec. 2018}
{Postdoctoral Research Assistant}
{\href{http://antennas.eecs.qmul.ac.uk/}{Queen Mary, University of London}, London, UK}
{Project AOTOMAT: Optimization tools and machine learning for the design of electromagnetic
devices and materials.
}

\entry
{Jul. 2014 \\Dec. 2016}
{Postdoctoral Research Assistant}
{\href{http://antennas.eecs.qmul.ac.uk/}{Queen Mary, University of London}, London, UK}
{\href{http://www.quest-spatial-transformation.org/}{Project Quest for Ultimate Electromagnetics Using Spatial Transformations (QUEST).}
Transformation Optics applied to the design, fabrication and characterization of novel electromagnetic devices using metamaterials.
Development of simulation tools and optimization techniques.
}


\entry
{Nov. 2013\\ Jan. 2014}
{Postdoctoral Research Assistant}
{\href{http://www.fresnel.fr/}{Institut Fresnel}, Marseille, France}
{
Numerical study of the coupling of light
to subwavelength resonant optical antennas and control of the local density of states.
}
%------------------------------------------------
\entry
{May 2013 \\Oct. 2013}
{Postdoctoral Research Assistant}
{\href{http://www.fresnel.fr/}{Institut Fresnel}, Marseille, France}
{
Development of simulation tools for ray tracing in complex media, inverse problem of
finding index distribution to make light follow a prescribed path, deshomogenization
technique with graded index photonic crystals.
}

%------------------------------------------------
\entry
{Oct. 2009\\Apr. 2013}
{PhD in Physics}
{\href{http://www.fresnel.fr/}{Institut Fresnel}~--~\href{http://www.silios.com/}{Silios Technologies}, Marseille, France}
{\emph{\href{http://tel.archives-ouvertes.fr/index.php?halsid=slas337fv1oqlj1okgkq7q42i5&view_this_doc=tel-00918651&version=1}
{Study of open electromagnetic resonators by modal approach.
Application to infrared multispectral filtering.}} (\emph{joint academia/industry funding})\\
FEM modelling of metamaterials, spectral analysis quasi-normal mode expansion.
Application to the design of infrared filters for multispectral imaging devices.
Fabrication and characterization reflexion bandcut and transmission bandpass filters.
}


%------------------------------------------------
% 
% \entry
% {Apr. 2009 \\Sept. 2009}
% {MS Research Internship}
% {\href{http://www.lma.cnrs-mrs.fr/}{Laboratoire de Mécanique et d'Acoustique, CNRS, Marseille, France}}
% {\emph{Model order reduction of reed instruments for sound synthesis: a proper orthogonal modes approach.}\\
% Numerical study and validation of proper orthogonal decomposition of the pressure field,
% optimal in the meaning of energy distribution.
% }


\end{entrylist}

%----------------------------------------------------------------------------------------
%	Teaching/supervising experience SECTION
%----------------------------------------------------------------------------------------

\vspace*{-0.2cm}
\section{Teaching/supervising experience}

\begin{entrylist}
%------------------------------------------------
\entry
{2011-2012}
{Internship supervisor}
{{Institut Fresnel}, CNRS, Centrale Marseille, Marseille, France}
{Optimization of diffractive spectral infrared filters (1 engineer student, 3 months).\\
Optimization of absorption in solar cells (4 engineer students, 1 month).
}

%------------------------------------------------
\entry
{2019}
{Teaching Assistant}
{\href{http://antennas.eecs.qmul.ac.uk/}{Queen Mary, University of London}, London, UK}
{Quantum Programming, lecture tutorials and coding laboratory (10 Master students, 6 months).}

%------------------------------------------------

\end{entrylist}

\vspace*{-0.2cm}
\section{Awards and honours}

{Best PhD thesis in 2013 award} from the {\href{http://ed352.sciences.univmed.fr/}{Doctoral School 352, Physics and Condensed Matter Science}}

{Best PhD thesis in 2013 award from {\href{www.cnano-paca.fr/}{CNano PACA}}, finalized research category}


\newgeometry{left=2cm,top=1cm,right=1.5cm,bottom=1.cm,nohead,nofoot}

 \section{Publications}
% % sorting=ydnt,
 \printbibsection{article}{Articles in peer-reviewed journal} % Print all articles from the bibliography

% \printbibsection{inproceedings}{international peer-reviewed conferences/proceedings}

%
\begin{refsection}

\nocite{*}
\printbibliography[type=inproceedings, title={Proceedings of international peer-reviewed conferences}, keyword={lecture comittee}, heading=subbibliography]
\end{refsection}

\printbibsection{book}{Contribution to book chapter} % Print all books from the bibliography


% \begin{refsection}
% \nocite{*}
% \printbibliography[type=inproceedings, title={International peer-reviewed conferences}, keyword={conference}, heading=subbibliography]
% \end{refsection}
%


\end{document}
