%%%%%%%%%%%%%%%%%%%%%%%%%%%%%%%%%%%%%%%%%

\documentclass[11pt]{academia}

\geometry{left=2cm,top=2cm,right=2cm,bottom=2cm,nohead,nofoot}

\def\firstname{Benjamin}
\def\familyname{Vial}
\def\FileSubject{research plan}
\def\FileAuthor{\firstname~\familyname}
\def\FileTitle{\firstname~\familyname's~\FileSubject}
\def\FileKeyWords{\firstname~\familyname, \FileSubject}

\RequirePackage[unicode]{hyperref}% unicode is required for unicode pdf metadata
\hypersetup{
    breaklinks,
    baseurl       = http://,
    pdfborder     = 0 0 0,
    pdfpagemode   = UseNone,% do not show thumbnails or bookmarks on opening
    pdfstartpage  = 1,
%    pdfproducer   = {\LaTeX{}},% will/should be set automatically to the correct TeX engine used
    bookmarksopen = true,
    bookmarksdepth= 2,% to show sections and subsections
    pdfauthor   = {\FileAuthor},%
    pdftitle    = {\FileTitle},%
    pdfsubject  = {\FileSubject},%
    pdfkeywords = {\FileKeyWords},%
    pdfcreator  = {\LaTeX},%
    pdfproducer = {\LaTeX}
}

\addbibresource{better.bib}
\addbibresource{~/.bib/biblio.bib}

\begin{document}

\header{Research}{~Plan}{Benjamin Vial~|~Application to the post of Lecturer in Quantum and Terahertz Experimental Electronic
Systems} 

\vspace*{1cm }
\newrefsection
%%%%%%%%%%%% Summary %%%%%%%%%%%%%%%%%%%%%%%%%
\section{Summary}


I will focus my research on three main areas, 
aligned with the expertise of the Antennas and Electromagnetic resarch group 
and related  to the School of Engineering and Computer Science aims for teaching. 
The first one is to develop THz research and experimental facilities and foster collaboration with 
colleagues in QMUL for biomedical applications. The second direction is aimed 
at studying and realizing reconfigurable antennas and metamaterials at 
microwave and THz frequencies. Those two areas will be supported by my 
experience in developing bespoke numerical schemes and optimization 
algorithms, including the development of inovative tools based on machine 
learning, which constitutes the third topic that I propose to pursue.


\vspace*{1cm }
%%%%%%%%%%%% THz and bio %%%%%%%%%%%%%%%%%%%%%%%%%

\section{Terahertz technologies for biomedical applications}


%%%%%%%%%%%%%%%%%%%%%%%%%%%% FIGURE %%%%%%%%%%%%%%%%%%%%%%%%%%%%%%%%%%%%%%%%%%
\wfig{bio}{(a) Biological applications of THz technologies. Typical THz-TDS imaging setup in transmission (b) and reflection (c).\label{fig:bio}}{0.5}{R}
%%%%%%%%%%%%%%%%%%%%%%%%%%%%%%%%%%%%%%%%%%%%%%%%%%%%%%%%%%%%%%%%%%%%%%%%%%%%%%

% ## Proteins
% ## Cells
Terahertz (THz) radiation generally refers to the frequency band spanning 0.1–10 THz, which lies between
the microwave and infrared regions of the electromagnetic spectrum. Due to the lack of
effective sources and detectors, this ‘THz gap’ remained
unexplored until advances in physics during the 1980s. 
Specifically, the rapid development of modern terahertz time-domain spectroscopy
(THz-TDS), has been widely utilized in applications such as materials
science, astronomy, microelectronics, and biomedical science\autocite{dhillon2017TerahertzScience2017}.
There has been great interest in applying THz spectroscopy to probe and characterize various 
biomaterials in recent decades because most low-frequency biomolecular motions 
lie in the same frequency range as THz radiation\autocite{sunRecentAdvancesTerahertz2017}.

I plan to foster existing collaboration between the group and colleagues 
at QMUL to study applications of THz science for biomedical research. 
Dr Akram Alomainy and Dr James Kelly, two lecturers in the Antenna
and Electromagnetics group, are likely to secure a grant to invest 
on a new fiber optics based THz-TDS that would improve on the existing facility 
in term of sensitivity and spectral resolution. 
In addition, Professor Mira Naftaly, a Senior Research Scientist at National Physical Laboratory, 
has recently joined the antenna group as a visiting academic and will 
surely provide her expertise and invaluable insights in terahertz technologies.



\subsection{Real time monitoring of proteins and biomaterial characterization}
Conformational changes, which are essential
for protein function, directly affect the dielectric response in the THz range (see Fig.~\ref{fig:bio}a). 
Due to its distinctive spectral responses to THz radiation, the dynamic hydration shell can
be precisely determined by probing protein-induced fast solvation dynamics by 
THz spectroscopy, and predicted by molecular dynamics simulations\autocite{sushkoTerahertzSpectralDomain2013}. 
Innovative measurements setups, design of waveplates for efficient 
measurements of circular dichroism\autocite{chengQuasiOpticalSubTHzCircular2020} 
and parameter extraction will be explored 
to help understand and monitor structural changes in biomolecules over time.\\
THz-TDS is an in situ non-destructive technique that has been used in QMUL to 
help understanding the setting mechanisms and associated dynamics of cementitious
materials\autocite{tianAtomicVibrationalOrigins2015}. 
This collaboration with Gregory Chass in the School of Physical and Chemical Sciences 
would benefit from further developments to test and characterize novel biomaterials for 
medical and dental applications as well as natural mineralisation processes.


\subsection{THz imaging}
A valuable tool for exploration of materials and biological samples is 
the capability to do imaging, and I plan to add this feature to the 
experimental facilities of the antenna laboratory. 
In particular, THz phase
imaging, an advanced imaging technology which combines the benefits of THz and commonly used phase
imaging techniques, has recently received significant attention, with both pulsed 
and continuous-wave (CW) imaging systems\autocite{wanTerahertzPhaseImaging2020}. Pulsed THz phase imaging is a coherent measurement, 
which includes terahertz pulsed imaging (TPI) based on femtosecond
laser and holographic imaging in the time domain, both allowing phase and amplitude 
information of the electric
field to be recorded. CW THz phase imaging is mainly based on digital holography, 
interferometry and ptychography. These systems can obtain the complex amplitude 
by capturing diffraction patterns and applying numerical reconstruction techniques. 
THz imaging systems typically operate as follows (see Fig.~\ref{fig:bio}b and c): the light from pulsed
femtosecond laser pumps a crystal to generate a broadband THz radiation. 
The scattered radiation (either in transmission or reflection mode) 
is mapped pixel by pixel using x-y
translation of the sample or the beam in the focal plane. 
Off-axis parabolic mirrors are used to focus the
generated THz pulse onto the sample and to collect the reflected or
transmitted pulse after interaction with the sample. 
For biomedical imaging, systems are mostly used in a reflective geometry
 because the biological samples typically have high water contents
 and are generally strongly absorptive. 
 I will study the feasibility of such imaging systems and apply for funding through EPSRC in close 
 interaction with the group in order to identify the needs and most suitable technology.
Another advance that I envision for biological applications is to enhance terahertz biomedical
imaging by introducing exogenous contrast agents, such as gold nanoparticles, 
which have been shown to improve contrast.

\subsection{Cells and tissue}
Early studies of the origin of contrast between healthy and
diseased tissue in the THz region focused on changes in
water content\autocite{yuPotentialTerahertzImaging2012}. 
However, water content differences are not only limited to differences between 
healthy and diseased tissues but also the type of tissue, 
and that structural changes in healthy cells were
responsible in part for the changes in the THz properties\autocite{syTerahertzSpectroscopyLiver2010}.
Computational methods and inverse modelling are needed to interpret the  
EM response of biological tissues and cells. 
Effective media approximations have been widely
used to estimate their effective permittivity. A more rigorous approach 
that I have successfully employed is the two scale convergence homogenization method
\autocite{guenneauHomogenizationThreedimensionalFinite2000}. In comparison
with analytical models such as
Maxwell-Garnett or Bruggeman theories, this approach is not limited to a
few canonical inclusions geometries, and can handle arbitrary topologies
and media with spatially varying properties, which will help modelling accurately 
the interaction of tissues with THz waves, such as 
human skin\autocite{wangTHzSensingHuman2021}. Full wave EM simulations 
and high frequency\autocite{crasterHighfrequencyHomogenizationPeriodic2010}
or high contrast\autocite{cherednichenkoHomogenizationSystemHighcontrast2015} 
homogenization techniques must be further developed, and fitting to a variety of clinical data, 
with the addition of inverse modelling, will help develop technology and 
aid better understanding of how terahertz interacts with tissues. 


\subsection{Enhanced detection with metamaterials and metasurfaces}

Several bottlenecks exist in the development of terahertz phase
imaging and detection. The lack of suitable beam shaping components to control the
profile of the object illumination beam affects image quality and 
the lack of terahertz imaging lenses with large numerical aperture makes it
difficult to collect high-frequency components and thus reduces resolution. 
Therefore, it is necessary to further develop high-performance
terahertz sources, detectors and imaging devices. I plan to investigate the 
design of imaging lenses at THz frequencies, in particular using my 
expertise in inverse design. The fabrication will be carried out with the group's
two-photon polymerization high-precision 3D printer (Nanoscribe), 
a state-of-the-art facility in additive manufacturing microfabrication. 
 

 \newpage
%%%%%%%%%%%% Tunable MTM and antennas %%%%%%%%%%%%%%%%%%%%%%%%%

\section{Reconfigurable metamaterials and antennas}


%%%%%%%%%%%%%%%%%%%%%%%%%%%% FIGURE %%%%%%%%%%%%%%%%%%%%%%%%%%%%%%%%%%%%%%%%%%
\wfig{ms}{Multi-scale view of ferroelectric materials and the
different length scales and associated theory and
modelling.\label{fig:ms}}{0.5}{R}
%%%%%%%%%%%%%%%%%%%%%%%%%%%%%%%%%%%%%%%%%%%%%%%%%%%%%%%%%%%%%%%%%%%%%%%%%%%%%%

Multifunctional and reconfigurable systems are increasingly demanded for
a wide range of applications in Electromagnetics and Photonics
\autocite{oliveriReconfigurableElectromagneticsMetamaterials2015}. 
In parallel, the development of artificial media and metamaterials, allowing unprecedented
control of electromagnetic (EM) waves by carefully engineered
subwavelength structures, has stimulated the research interest in
tunable materials.  I will focus 
primarily on ferroelectric materials
play a crucial role in microwave applications needing reconfigurability
\autocite{tagantsevFerroelectricMaterialsMicrowave2018} (such as antenna beam steering, phase shifters, filters, and tunable power
splitters) but plan to extend my research to other type of tuning 
strategies, such as phase change materials, liquid crystals, graphene or 
mechanically adjustable elements. Due to their inherent
multi-physics and multi-scale behaviour (see Fig.~\ref{fig:ms}),
ferroelectrics need improved theoretical, numerical and experimental 
tools as well as a proper linking of the different length scales in order to advance
their understanding and guide their synthesis for use in microwave
devices.

%%%%%%%%%%%%%%%%%%%%%%%%%%%% FIGURE %%%%%%%%%%%%%%%%%%%%%%%%%%%%%%%%%%%%%%%%%%
\wfig{msthz}{BST metasurface composed of micro-pillars on thin film. (a) Extracted BST permitivity (left: real part, right: imaginary part) from THz-TDS transmission measurement of the film alone. (b) measured and simulated with the Fourier modal method (FMM) transmission through the metasurface.\label{fig:bstms}}{0.5}{L}
%%%%%%%%%%%%%%%%%%%%%%%%%%%%%%%%%%%%%%%%%%%%%%%%%%%%%%%%%%%%%%%%%%%%%%%%%%%%%%

\hypertarget{sec:meta}{%
\subsection{Ferroelectric metasurfaces}\label{sec:ferrometasurf}}

Structured BST film with micro-arrays have been fabricated by transfer printing 
in the School of Material Science at QMUL (cf. Fig.~\ref{fig:bstms}). 
Thin BST films deposited on silicon substrate where characterized by 
THz-TDS and the permittivity extracted (cf. Fig.~(\ref{fig:bstms}a). Transmission through the metasurface 
agrees well with numerical simulation with the Fourier Modal Method (FMM) where 
the measured permittivity of the substrate and BST were included (see Fig.~\ref{fig:bstms}b). 
With those encouraging preliminary results, I will continue 
this area of research and help improve, characterize and test different 
BST and ferroelectric materials at THz frequencies. Geometric optimization of the 
microarray will be carried out to obtain functionalities such as beam steering, 
lensing or filtering. Change in permittivity due to an applied voltage or temperature change 
must be measured and will be used in further design of tunable ferroelectric metadevices. 
In order to extract parameters from transmission measurements, I have developed an open-source Python 
package\autocite{tdsxtract} 
with tools for pre and post processing 
raw experimental data, and obtain the permittivity of sample through minimization 
of the error between measured and analytical transmission coefficients. This 
will be further improved and can readily be applied to arbitrary incident 
angle and polarization, and would be easily extended to multilayer stacks with 
several layers, reflection mode operation, and anisotropic and magnetic materials.




\hypertarget{sec:dft}{%
\subsection{Atomistic modelling of ferroelectrics using Density
Functional Theory (DFT)}\label{sec:dft}}

\emph{Ab initio} simulation of ferroelectic materials from their
atomistic arrangement will be carried out to identify properties and new
components with improved performances. Structural
composition, proportions of atoms (e.g.~in Barium strontium titanate
(BST) \({\rm Ba}_{x}{\rm Sr}_{1-x}{\rm TiO}_3\) with a barium ratio of
\(x\)), oxygen vacancies, can be studied by such models. In addition,
the effect of external stimuli such as applied voltage, mechanical
strain can be included, whilst the temperature dependence in DFT can be
added \emph{a posteriori}. The primary focus will be on the frequency
dependence of permittivity and tunability, as it will affect the final
design of devices in the radiofrequency and THz spectrum. Along with \emph{in
silico} new material discovery through computational material science,
experimental exploration of the new ferroelectrics needs to be done to
verify and inform the modeller.

\hypertarget{sec:pf}{%
\subsection{Phase field modelling of ferroelectric
domains}\label{sec:pf}}

The electromechanical properties of ferroelectrics can be studied by the
phase field method \autocite{suContinuumThermodynamicsFerroelectric2007},
where one considers an order parameter (usually the polarization vector)
which varies continuously. Hence domain patterns with uniform
polarization are separated by domain walls where the polarization
switches smoothly. This intermediate mesoscale modelling is necessary
and relevant as the effective properties of ceramics are often dependant
of the domain arrangement and grain boundaries. Polycrystalline
ferroelectrics, the effect of domain grain size, distribution,
boundaries thickness, charge defects and structural inclusions, can be
studied as a function of electromechanical loading, allowing to
reproduce P-E loops from which one can extract the tunability and
directly compare with experiments. The free energy functional used in
phase field models has to be adjusted to fit the properties of single
domains, and the coefficients can be obtained from DFT simulations and/or measurements
\autocite{volkerMultiscaleModelingFerroelectric2011}. 


\hypertarget{sec:meta}{%
\subsection{Metamaterials and homogenization}\label{sec:meta}}

I will continue the study of effective properties of ferroelectric
metamaterials with subwavelength features for enhanced tunability
composites \autocite{vialEnhancedTunabilityFerroelectric2019, vialHighFrequencyMetaferroelectrics2021}. 
To enable the fabrication of those metaceramics and the measurement of their
properties at DC, microwave and THz frequencies, I will
identify relevant length scale compatible with our fabrication and
measurement capabilities, including the newly acquired 3D printer 
with ceramic-based resins, as well as electrode layout for biasing the samples.
The study of properties at the mesoscale from phase field simulations
will be included in the homogenization.

\vspace*{1cm}
%%%%%%%%%%%% CEM and ML%%%%%%%%%%%%%%%%%%%%%%%%%

\section{Inverse design and data driven computational electromagnetism}

I believe my expertise in numerical simulation and optimization algorithms 
would benefit the antenna group and will be a key to the success of the 
different research projects that I plan to undertake. I have developed bespoke 
open-source numerical tools employed in Electromagnetics, Photonics 
and antenna design that, together with commercial software, will help in 
designing novel devices and interpreting experimental results: 
(i) a versatile finite element based code\autocite{gyptis} with mesh generation, solving and post-processing 
complex electromagnetic problems; (ii) an 
implementation of the Fourier Modal Method\autocite{nannos} (also known as Rigorous Coupled Wave Analysis) 
with various formulations for the simulation of periodic stratified media 
such as frequency selective surfaces, metamaterials, photonic crystal slabs 
or diffraction gratings. 
Both codes have built-in automatic adjoint calculation capabilities 
that makes them suitable for gradient-based optimization.

%%%%%%%%%%%%%%%%%%%%%%%%%%%% FIGURE %%%%%%%%%%%%%%%%%%%%%%%%%%%%%%%%%%%%%%%%%%
\wfig{lens}{Inverse design of a multi-beam lens antenna. Permittivities: material 0:  $\varepsilon = 3$, material 1:  $\varepsilon = 6.5$, $\tan\delta = 0.004$. Lens size =  $1\lambda \times 6 \lambda$. Source - lens distance = $3\lambda$. Source – ground plane distance = $\lambda/4$. The objective is to maximize the directivity for two target angles : 0 and 30 degrees.\label{fig:lens}}{0.4}{R}
%%%%%%%%%%%%%%%%%%%%%%%%%%%%%%%%%%


\subsection{Topology optimization}
% 
% The development of photonics and electromagnetic devices with
% characteristic size of the order of the wavelength has historically
% relied on trial and error approaches. Starting from an a given physical
% process, a small set of key parameters is tuned to achieve an acceptable
% level of matching with a predefined figure of merit required by the
% application. This intuition based method has helped to develop a diverse
% and extensively used collection of designs, taking advantages of
% photonic resonances, dispersion engineering, waveguiding or antenna
% radiation principles enabling an increasingly finer 
% control of light accross the electromagnetic spectrum.\\


%%%%%%%%%%%%%%%%%%%%%%%%%%%%%%%%%%%%%%%%%%%%

In the past two decades, topology optimization (TO)
\autocite{bendsoeTopologyOptimizationTheory2013} has become a widely used
tool in computational electromagnetism
\autocite{moleskyInverseDesignNanophotonics2018} and has allowed the
inverse design of a broad range of devices such as invisibility cloaks
illusion devices\autocite{vialOptimizedMicrowaveIllusion2017},
metasurfaces,
photonic crystals
and metamaterials,
to name a few. Broadly speaking, density based TO is an inverse design
procedure that can produce highly optimized structures serving a
dedicated objective. One of its main advantage is to offer unparalleled
design freedom since the material distribution is updated locally (at
the pixel or voxel level) inside the domain of interest. On the other
hand, fabrication constraints often limit this versatility and several
auxiliary tools can be included to tackle those issues, for instance for
imposing minimal length-scales or ensuring the connectivity of the
resulting layout. Since the number of degrees of freedom is usually
prohibitively high to obtain the gradient using naive finite
differences, adjoint sensitivity analysis
\autocite{jensenTopologyOptimizationNanophotonics2010} is an indispensable
part of all inverse design algorithms. 
During my work in QMUL, I have developed general tools for inverse design of 
materials and devices. I plan to continue to improve those and apply them 
to the design of antenna devices and metamaterials, that are related with 
the rest of my research plans and offer my expertise to the group. 
One area of improvement is the inclusion of metallic parts 
with density based parametrization schemes, which show poor convergence 
because of the high contrast in material conductivities\autocite{aageTopologyOptimizationMetallic2010}. 
Another direction is to explore another strategy based on level-set functions 
that remove the unphysical intermediate material properties. 
Multi-objective optimization and parallelization on the school computing 
cluster are already implemented. This design technique allows the production 
of optimized parts with improved performances with often non intuitive, 
complex layouts, which are particularly suited for 3D printing. 
As an example of application, I include preliminary results on an optimized 
metalens for beam shaping (cf. Fig.~\ref{fig:lens}).

\subsection{Machine learning}


%%%%%%%%%%%%%%%%%%%%%%%%%%%% FIGURE %%%%%%%%%%%%%%%%%%%%%%%%%%%%%%%%%%%%%%%%%%
\wfig{ml}{Machine learning for electromagnetic design (a) Dielectric metasurface reflection spectra prediction (100 000 points in dataset, square unit cell with period=20mm made of two dielectrics with permittivity 3.0 ("0") and 4.5 ("1") and loss tangent 0.001, thickness=5mm). (b) Photonic crystal eigenmode prediction(100 000 points in dataset, input: permittivity pattern (2D, 32x32), output: 6 first eigenvalues/eigenmodes at three symmetry points of the first Brillouin zone.).\label{fig:ml}}{0.4}{L}
%%%%%%%%%%%%%%%%%%%%%%%%%%%%%%%%%%%%%%%%%%%%%%%%%%%%%%%%%%%%%%%%%%%%%%%%%%%%%%

Recent advances in experimental and computational methods are increasing the quantity
and complexity of generated data. Innovative approaches and tools play
an important role in shaping design, characterization and optimization for the field of
electromagnetism. In particular, deep learning offers an efficient means to design photonic
structures\autocite{maDeepLearningDesign2020}, complementing conventional physics 
and rule-based methods with data-driven paradigms. My objective is to use various machine 
learning algorithms to predict antenna and metamaterial responses from 
a dataset of computer simulations. Once the neural network is trained, one obtains 
a fast reduced order model that can be employed to design and optimize novel EM devices 
very efficiently. For instance, I have developed a surrogate based global optimization 
code (based on a genetic algorithm), where computationally 
expensive objective function evaluations are partly replaced by a learned model. 
As an example, I have recently applied deep convolutional 
neural networks for predicting the reflection response from a dielectric metasurface 
at millimetre waves (see Fig.~\ref{fig:ml}a), which I plan to extend by 
training the model on both amplitude and phase data, and include different polarization 
of the incident wave, and use this fast model for optimization of spectral filters using 
a layered approach by stacking multiple permittivity patterns. 
Other recent results include the prediction of eigenvalues and eigenmodes of 2D 
photonic crystals (cf. Fig.~\ref{fig:ml}b). Eventually, once the model is trained
 one can predict quickly the eigenpairs at the symmetry points of the first Brillouin zone, 
 and I plan to implement a reduced Bloch mode expansion\autocite{husseinReducedBlochMode2009} for fast band diagram and 
 isofrequency contour calculations, so the resulting efficient simulation tool 
 can be used for exploring and optimizing exotic physical effects such as dispersion engineering, hyperbolic media, maximized band-gap structures, exceptional and topological Photonics.
  
% %%%%%%%%%%%% Quantum %%%%%%%%%%%%%%%%%%%%%%%%%
% 
% \section{Microwave engineering for quantum computing}
% 
% In the last decade, quantum computing has grown from 
% a mostly theoretical field to one that could potentially 
% reshape computing as we know it. Several major industrial
% research efforts\autocite{aruteQuantumSupremacyUsing2019}, 
% together with large national research programmes (such as the 
% UK National Quantum Technologies Programme\autocite{knightUKNationalQuantum2019}) 
% worldwide are boosting this unique and emerging technological area, from 
% dedicated software implementation to the design of devices necessary 
% to built a working quantum computer. 
% However, because quantum computing is highly interdisciplinary, 
% there is currently a need for engineers who understand both the 
% quantum foundations and microwave engineering aspects necessary to 
% build a working quantum computer. \\
% The fundamental information carriers in a quantum computer
% are quantum bits, or qubits, in analogy to the logical bits used
% in a classical computer. Due to quantum superposition, 
% the $2^N$-dimensional state space of $N$ qubits can hold ex-
% ponentially more information than that of $N$ classical bits,
% offering the hope of greatly increased computing power, with the caveat 
% that at the end result of a computation, the measured state will 
% collapse to a single configuration due to entanglement.\\
% Although many hardware platforms are still being pursued to implement quantum information
% processing, including trapped ions,
% bulk and integrated photonics, cold atoms,
% semiconductor spin qubits, superconducting circuit architectures are one of the
% leading candidates\autocite{blaisQuantuminformationProcessingCircuit2007, 
% voolIntroductionQuantumElectromagnetic2017,guMicrowavePhotonicsSuperconducting2017}. 
% These systems, also often called
% circuit quantum electrodynamics (cQED) devices
% Numerous research and technological challenges needs to 
% be overcome to the scaling of quantum computing with such systems, also often called
% circuit quantum electrodynamics (cQED) devices, and I plan to contribute 
% to the advancement of.
% Of the many superconducting qubits that have been designed
% to date, the transmon has become one of the most widely
% used since its creation and forms a key part of many scalable
% quantum information processing architectures using superconducting circuits. 
% Fundamentally, one way to think of a qubit is as a high-quality-factor 
% electromagnetic resonator, but unlike ordinary linear resonators, 
% qubits are extremely anharmonic (or nonlinear).
% 
% 
% Beyond the classical to quantum interface, there are signif-
% icant microwave challenges in the design of the quantum
% processor itself. For instance, considering superconducting
% qubits, as the dimensions of the quantum processor grow,
% designers will have to rely more heavily on electromagnetic
% simulation tools to predict and avoid undesired moding in
% what will eventually become wafer-scale devices. Modeling
% these effects requires incorporating superconducting physics
% and the cryogenic properties of dielectrics into electromag-
% netic tools while developing methods to efficiently solve for
% high-Q resonances in large structures. Microwave expertise
% will be required, both in developing efficient simulation tools
% tailored to this purpose as well as in developing techniques to
% mitigate these undesired modes.

% 
% Engage undergraduate students taking the Quantum Computation and Communication 
% module as part of their research project.


% 
% Next, we can consider the quantum chip itself. The
% typical pitch of 2D grids of transmon qubits is approxi-
% mately 1 mm, so a notional grid of 1 million qubits might
% therefore cover a square measuring 1 m on a side. Such
% a device will not be composed of a single silicon chip.
% Instead, many smaller chips will need to be stitched
% together. Multichip assembly and 3D integration tech-
% nologies will have to be adapted to the material sets,
% loss and crosstalk requirements, cryogenic operation,
% and layout constraints imposed by this application. The
% chip-to-chip interconnect will need to have exquisitely
% low loss and low crosstalk while maintaining the cou-
% pling strength needed to do fast qubit–qubit logic gates.
% Electromagnetic, thermal, and mechanical simulations
% of the packaged assembly will become crucial. Parasitic
% resonances and package modes in assemblies of the
% envisioned size could overlap the qubit frequencies, and
% their coupling to the qubits will have to be understood
% and mitigated. These are just a few of the many areas in
% which microwave engineering will be required in the
% quest to implement a fault-tolerant quantum computer.
% As the systems progress, it will be essential that more
% microwave engineers engage in these effort



% - https://uknqt.ukri.org/about/nqcc/

% %%%%%%%%%%%% Track record %%%%%%%%%%%%%%%%%%%%%%%%%
% \section{Track record}
% \kant[2-4]
% 
% 
% 
% \newrefcontext[labelprefix=TR]
% \printbibliography[title={Recent related publications from the team}]
% 
% 
% %%%%%%%%%%%% Main body %%%%%%%%%%%%%%%%%%%%%%%%%
% \newrefsection
% \section{Research plan}
% 
% %% command for wrapped figure: 
% %% \wfig{image_file}{cpation}{size}{position}
% %% size between 0 and 1 (proportion of the page width
% %% position is either right (R) or left (L)
% % \wfig{img}{Caption for this figure\label{fig:1}}{0.5}{R}
% 
% 
% \wfig{drawing}{Caption for this figure\label{fig:2}}{0.5}{R}
% 
% 
% 
% \subsection{Aims and Objectives}
% \kant[3]
%  Also see \fig{fig:1}, and some citations\autocite{foster2017beam, Vial2017, vial2016class} to appear at the end.\\
% 
% \workpackage{Title of this work package\label{wp:1}}{YH}{MR, BV}
% \kant[4]
% \task{This is the description of a task}
% \task{This is the description of a second task}
% 
% Refer to WP~\ref{wp:1}.
% 
% 
% \subsection{Expected Outcomes and Results}
% \kant[5-6]
% 
% \wfig{drawing}{Caption for this figure\label{fig:1}}{0.4}{L}
% 
% \subsection{National Importance}
% \kant[6]

\printbibliography[title={References}]

 
\end{document}
