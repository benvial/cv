

\documentclass{cv}

\geometry{left=6.0cm,top=1.8cm,right=1.5cm,bottom=0.7cm,nohead,nofoot}


\def\firstname{Benjamin}
\def\familyname{Vial}
\def\FileSubject{curriculum vitae}
\def\FileAuthor{\firstname~\familyname}
\def\FileTitle{\firstname~\familyname's~\FileSubject}
\def\FileKeyWords{\firstname~\familyname, \FileSubject}


\RequirePackage[unicode]{hyperref}% unicode is required for unicode pdf metadata
\hypersetup{
	breaklinks,
	baseurl       = http://,
	pdfborder     = 0 0 0,
	pdfpagemode   = UseNone,% do not show thumbnails or bookmarks on opening
	pdfstartpage  = 1,
	bookmarksopen = true,
	bookmarksdepth= 2,% to show sections and subsections
	pdfauthor   = {\FileAuthor},%
	pdftitle    = {\FileTitle},%
	pdfsubject  = {\FileSubject},%
	pdfkeywords = {\FileKeyWords},%
	pdfcreator  = {\LaTeX},
}



\addbibresource{better.bib} % Specify the bibliography file to include publications

\begin{document}

\header{Benjamin}{~Vial}{Docteur Ingénieur | Optique, Photonique et micro-ondes} 

%----------------------------------------------------------------------------------------
%	SIDEBAR SECTION
%----------------------------------------------------------------------------------------

\begin{aside} % In the aside, each new line forces a line break
	\section{Contact}
	\faHome~~146 Glyn road
	London E5 0JE, UK
	\faPhone~~+44~7840~029~744
	\faEnvelope~~\href{mailto:b.vial@qmul.ac.uk}{b.vial@qmul.ac.uk}
	\faUser~~\href{http://bvial.info/}{bvial.info}
	\section{Informations}
	né le 09/11/1984
	nationalité française
	\section{Langues}
	français : langue maternelle
	anglais : courant
	espagnol : bases
	\section{Programmation}
	\textbf{systèmes d'exploitation}
	Linux, Windows
	\textbf{languages et scripts}
	Python, Matlab, Mathematica, \LaTeX, C, C++, Q\#, HTML, CSS
	\textbf{applications}
	git, Comsol Multiphysics, Fenics, Gmsh, GetDP, Gimp, LibreOffice, Labview
	\section{Intérêts}
	\textbf{professionels}
	Photonique
	Optique
	ingéniérie micro-ondes
	processus résonants
	interaction lumière matière
	analyse modale
	domaine THz
	métamatériaux et métasurfaces
	Optique de Transformation
	modélisation numérique
	méthode des éléments finis
	methode modale de Fourier
	FDTD
	techniques d'optimization
	problèmes inverse
	apprentissage machine
	physique des ondes
	fabrication
	caracterisation
	science ouverte
	\textbf{personnels}
	jouer de la guitare, musique
	football, marche à pied
	voyages, cuisine
\end{aside}


%----------------------------------------------------------------------------------------
%	EDUCATION SECTION
%----------------------------------------------------------------------------------------


% \vspace*{-0.6cm}

\section{Formation}

\begin{entrylist}
%------------------------------------------------
\entry
{Apr. 2013}
{Thèse de doctorat {\normalfont en Physique}}
{\href{http://www.fresnel.fr/spip/}{Institut Fresnel}, CNRS, Centrale Marseille, Aix Marseille Universit\'e, Marseille, France}
{Optique, Photonique et traitement d'image}
%------------------------------------------------
\entry
{Oct. 2009}
{Master {\normalfont en Physique}}
{\href{http://www.centrale-marseille.fr/}{Centrale Marseille}~--~
\href{http://www.lma.cnrs-mrs.fr/}{Laboratoire de Mécanique et d'Acoustique}, CNRS, Marseille, France}
{Mécanique, Physique et Ingéniérie, specialisation en Acoustique}

\entry
{Oct. 2009}
{Diplôme d'{Ingénieur Généraliste}}
{\href{http://www.centrale-marseille.fr/}{Centrale Marseille}, Marseille, France}
{Formation scientifique généraliste}
%------------------------------------------------
\end{entrylist}


%----------------------------------------------------------------------------------------
%	WORK EXPERIENCE SECTION
%----------------------------------------------------------------------------------------
\vspace*{-0.2cm}
\section{Activités de recherche}

\begin{entrylist}
	%------------------------------------------------



	\entry
	{depuis\\Jan. 2019}
	{Postdoctorat}
	{\href{http://antennas.eecs.qmul.ac.uk/}{Queen Mary, University of London}, London, UK}
	{\href{https://animate-research.com/}{Projet ANIMATE}: couplage non-linéaire et propriétés effectives de métamatériaux ferroélectriques, conception inverse pour maximiser leur ajustabilité, 
	cractérisation de matériaux dans les domaines micro-ondes et THz.
	}


	\entry
	{Jan. 2017 \\Déc. 2018}
	{Postdoctorat}
	{\href{http://antennas.eecs.qmul.ac.uk/}{Queen Mary, University of London}, London, UK}
	{Projet AOTOMAT: Outils d'optimisation pour la conception de matériaux et composants electromagnétiques.
	}

	\entry
	{Juil. 2014 \\Déc. 2016}
	{Postdoctorat}
	{\href{http://antennas.eecs.qmul.ac.uk/}{Queen Mary, University of London}, London, UK}
	{\href{http://www.quest-spatial-transformation.org/}{Projet QUEST: Quest for Ultimate Electromagnetics Using Spatial Transformations.}
	Optique de Transformation appliquée à la conception, l'analyse et la caractérisation de composants à base de métamatériaux.
	}


	\entry
	{Nov. 2013\\ Jan. 2014}
	{Postdoctorat}
	{\href{http://www.fresnel.fr/}{Institut Fresnel}, Marseille, France}
	{
	Antennes résonantes. \'Etude numérique du couplage entre lumière et particules sub longueur d'onde.
	Analyse modale des résonances électriques et magnétiques pour contrôler l'émission et la densité locale d'états.
	}
	%------------------------------------------------
	\entry
	{May 2013 \\Oct. 2013}
	{Postdoctorat}
	{\href{http://www.fresnel.fr/}{Institut Fresnel}, Marseille, France}
	{
		Développement d'outils de simulation pour le tracé de rayons en milieu complexe.
	}

	%------------------------------------------------
	\entry
	{Oct. 2009\\Avr. 2013}
	{Thèse de Doctorat en Physique}
	{\href{http://www.fresnel.fr/}{Institut Fresnel}~--~\href{http://www.silios.com/}{Silios Technologies}, Marseille, France}
	{\emph{\href{http://tel.archives-ouvertes.fr/index.php?halsid=slas337fv1oqlj1okgkq7q42i5&view_this_doc=tel-00918651&version=1}
			{\'Etude de résonateurs électromagnétiques ouverts par approche modale.
			Application au filtrage multispectral dans l'infrarouge.}} (\emph{mi-temps université/entreprise})\\
			Modèles numérique part éléments finis pour le calcul des modes propres de
			 l'opérateur de Maxwell pour des structures ouvertes et décomposition modale sur modes quasi-normaux.
			  Application à la conception, fabrication et caractérisation de
			  filtres diffractifs dans l'infrarouge pour des imageurs multispectraux: passe bande en
			   transmission et coupe bande en réflexion basés sur des métamatériaux.
	}

\end{entrylist}

%----------------------------------------------------------------------------------------
%	Teaching/supervising experience SECTION
%----------------------------------------------------------------------------------------

\vspace*{-0.2cm}
\section{Expérience d'enseignement/encadrement}

\begin{entrylist}
	%------------------------------------------------
	\entry
	{2011-2012}
	{Tuteur de stage}
	{{Institut Fresnel}, CNRS, Centrale Marseille, Marseille, France}
	{Optimisation de filtres spectraux diffractifs (1 élève ingénieur, 6 mois).\\
		Optimisation de l'absorption dans des cellules solaires (4 élèves ingénieur, 3 mois).
	}

	%------------------------------------------------
	\entry
	{2019}
	{Assistant d'Enseignement}
	{\href{http://antennas.eecs.qmul.ac.uk/}{Queen Mary, University of London}, London, UK}
	{Informatique quantique. Cours sur les portes et circuits quantiques. 
	TDs sur ordinateur et projets de fin d'année et language de programmation Q\# et Python (10 élèves de Master, 6 mois).}

	%------------------------------------------------

\end{entrylist}

\vspace*{-0.2cm}
\section{Prix et récompenses}

  {Prix de la meilleure thèse 2014} {\href{https://ecole-doctorale-352.univ-amu.fr/en}{\'Ecole Doctorale 352, Physique et Science de la Matière}}

  {Prix de la meilleure thèse 2014 {\href{https://www.cnano-paca.fr/index.php?option=com_content&view=article&id=80}{CNano PACA}}, catégorie recherche finalisée}


\newgeometry{left=1.5cm,top=1.8cm,right=1.5cm,bottom=0.7cm,nohead,nofoot}



%----------------------------------------------------------------------------------------
%	Referees contact SECTION
%----------------------------------------------------------------------------------------


\section{Références}

\begin{minipage}{.6\textwidth}
	\textbf{Prof Yang Hao}\\
	\faHome~~School of Electronic Engineering and Computer Science\\
	Queen Mary University of London\\
	Peter Landin Building, 10 Godward Square, Mile End Road\\
	London E1 4FZ, United Kingdom\\
	\faEnvelope~~\href{mailto:y.hao@qmul.ac.uk}{y.hao@qmul.ac.uk}\\
	\faPhone~~+44 20 7882 5341\\
	\faUser~~\href{http://www.eecs.qmul.ac.uk/~yang/}{www.eecs.qmul.ac.uk/~yang/}
\end{minipage}%
\begin{minipage}{0.4\textwidth}
	\textbf{Prof Andr\'e Nicolet}\\
	\faHome~~Aix-Marseille Université\\
	Institut Fresnel (UMR CNRS 6133)\\
	Domaine Universitaire de Saint-Jérôme\\
	F13397 Marseille cedex 20, France\\
	\faEnvelope~~\href{mailto:andre.nicolet@fresnel.fr}{andre.nicolet@fresnel.fr}\\
	\faPhone~~+33 4 91 28 87 73\\
	\faUser~~\href{https://www.fresnel.fr/perso/nicolet/}{www.fresnel.fr/perso/nicolet/}
\end{minipage}


\vspace*{0.5cm}

%----------------------------------------------------------------------------------------
%	Publications SECTION
%----------------------------------------------------------------------------------------


\section{Publications}

\printbibsection{article}{Articles dans des revues internationales à comité de lecture}

\printbibsection{book}{Contribution à un chapitre de livre}

\printbibsectionkey{inproceedings}{Conférences internationales avec actes}{lecture comittee}

\printbibsectionkey{inproceedings}{Conférences internationales sans actes}{conference}

\printbibsectionkey{misc}{Thèse de doctorat}{phdthesis}

\printbibsectionkey{misc}{Codes et logiciels libres}{code}

\printbibsectionkey{misc}{En préparation}{prep}


\end{document}
